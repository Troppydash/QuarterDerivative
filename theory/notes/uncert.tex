\documentclass[a4paper,11pt]{article}

\usepackage{mathtools}
\usepackage{geometry}
\usepackage{parskip}
\usepackage{amssymb}

\usepackage{bm}

\DeclareMathOperator{\var}{\operatorname{Var}}
\DeclareMathOperator{\cov}{\operatorname{Cov}}
\DeclareMathOperator{\uncertabs}{\Delta}
\DeclareMathOperator{\uncertrel}{\%}
\DeclareMathOperator{\ex}{\operatorname{E}}

\newcommand{\pde}[2]{\frac{\partial {#1}}{\partial {#2}} }
\newcommand{\lst}[1]{({#1}_1, {#1}_2, \dots, {#1}_N)}

\begin{document}

\section{Definitions}
For a random variable $X$, define the absolute and relative uncertainty to be its standard deviations and relative standard deviations
\begin{align*}
    \uncertabs X &= \sqrt{\var(X)}\\
    \uncertrel X &= \left| \frac{\uncertabs X}{X} \right| = \left|\frac{\sqrt{\var(X)}}{X} \right|
\end{align*}
these values are fixed for a given $X$ with some distribution.

The realizations of these variables based on some sample (where the variance $\var(X)$ or value $X$ are sample values $x$ and sample standard deviations $s_X$) are thus
\begin{align*}
    \uncertabs x &= \sqrt{s_X}\\
    \uncertrel x &= \left|  \frac{\sqrt{s_X}}{x} \right|
\end{align*}
these values are random depending on the samples (a sampling distribution), with hopes that their expected values match the true uncertainties:
\begin{align*}
    \ex(\uncertabs x) &= \uncertabs X\\
    \ex(\uncertrel x) &= \uncertrel X
\end{align*}

Some identities that are relevant for uncertainties are
\begin{align*}
    \var X &= (\uncertabs X)^2\\
    &= X^2 (\uncertrel X)^2\\
    \var (aX + bY + c) &= |a|^2 \var(X) + |b|^2 \var(Y) \qquad \forall a,b,c\in \mathbb{R}
\end{align*}
for random variables $X$ and $Y$ that are independent\footnote{Implying that $\cov(X,Y)=r \sqrt{\var(X)\var(Y)} = 0$, where $r$ is the correlation coefficient}.

\newpage

\section{IB Formulas}
For functions of kind
\[
    f(X, Y) = X + Y
\]
we have that
\[
    \uncertabs f = \uncertabs X + \uncertabs Y
\]

For functions of kind
\[
    f(X, Y) = XY
\]
we have that
\[
    \uncertrel f = \uncertrel X + \uncertrel Y
\]

For functions of kind
\[
    f(X) = X^a \qquad \forall a \in \mathbb{R}
\]
we have that
\[
    \uncertrel f = |n| (\uncertrel X)
\]

\newpage
\section{First Order Methods}
For formulas, I'll write the first expression in vector notation, and the second in the expanded notation. The derivations are done in vectorized notations only.

For a general function
\[
    Y = f(\bm{X}) = f(X_1, X_2, \dots, X_N)
\]
consider the first-order taylor series expansion at $\bm{x} = (x_1, x_2, \dots, x_N)$, the realizations/measurements of the random variables:
\begin{align*}
    Y &\approx f(\bm{x}) + \sum_i^N (\bm{X}_i - \bm{x}_i) \frac{\partial f}{\partial \bm{X}_i} (\bm{x}) \\
    &\approx f(x_1, x_2, \dots, x_N) + \sum_{i=1}^N (X_i - x_i)\frac{\partial f}{\partial X_i} (x_1, x_2, \dots, x_N)
\end{align*}

Note that the value $\bm{x} = (x_1, x_2, \dots, x_N)$ are fixed par our experimental measurement, while $\bm{X} = (X_1, X_2, \dots, X_N)$ is a random variable with some distribution.

Taking variance on both sides, we have that
\begin{align*}
    \var(Y) &\approx \var \left(f(\bm{x}) + \sum_i^N (\bm{X}_i - \bm{x}_i) \frac{\partial f}{\partial \bm{X}_i} (\bm{x})\right)\\
    &\approx \sum_i^N \var \left((\bm{X}_i - \bm{x}_i) \frac{\partial f}{\partial \bm{X}_i} (\bm{x})
    \right)\\
    &\approx \sum_i^N \var (\bm{X}_i - \bm{x}_i) \left| \frac{\partial f}{\partial \bm{X}_i} (\bm{x})\right|^2\\
    &\approx \sum_i^N \var (\bm{X}_i) \left| \frac{\partial f}{\partial \bm{X}_i} (\bm{x})\right|^2\\
    (\uncertabs Y)^2 &\approx \sum_i^N (\uncertabs X_i)^2 \left| \frac{\partial f}{\partial \bm{X}_i} (\bm{x})\right|^2
\end{align*}

So, in general for the realizations on the values $\bm{x} = (x_1, x_2, \dots, x_N)$ and their uncertainties $\uncertabs \bm{x} = (\uncertabs x_1, \uncertabs x_2, \dots, \uncertabs x_N)$, the uncertainty in $y$ is
\begin{align*}
    (\uncertabs y)^2 &\approx \sum_i^N (\uncertabs \bm{x}_i)^2 \left| \frac{\partial f}{\partial \bm{X}_i} (\bm{x})\right|^2\\
    &\approx \sum_{i=1}^N (\uncertabs x_i)^2 \left| \frac{\partial f}{\partial X_i} (x_1, x_2, \dots, x_N)\right|^2
\end{align*}

We'll often use the realizations represented in lower-cased letters like $x$ and $\uncertabs x$, in place of the true random variables like $X$ and $\uncertabs X$ for the formulas.

\subsection{First-order Propagation Formulas}
The basic addition formula has
\[
    f(x, y) = x + y
\]
with
\[
    \pde{f}{x} = \pde{f}{y} = 1
\]

The uncertainty in $f$ is therefore
\[
    (\uncertabs f)^2 = (\uncertabs x)^2 + (\uncertabs y)^2
\]
of which resembles the IB formulas when $\uncertabs y \gg \uncertabs x$.

It may be interesting to model the residuals between the IB and first-order formulas by parameterizing $\uncertabs x$ on the line
\[
    \uncertabs x + \uncertabs y = r
\]
for a fixed number $r$.

\subsection{Limitations and Extensions}
The first order equations are limited in accuracy as the uncertainties grow larger. We can derive the second-order propagations with a bit of help by finding the distribution of $\bm{X}_i^2$ and assuming that they are all normally distributed.

\newpage
\section{Probability Methods}

For the probability method, we assume that the measurements are normally distributed. The propagation needs only to test the distribution of the function after the transformation.

We assume that all measurements $\bm{X}= (X_1, X_2, \dots, X_N)$ are normally distributed
\[
    \bm{X}_i \sim N(\bm{x}_i, \uncertabs \bm{x}_i )
\]
Notice that the uncertainties are the standard deviations, and observations are the means.

This method allows an arbitrary distribution of the measurement variables while ignoring all covariances in-play.

\subsection{Monte-Carlo Approach}
For a transformation
\[
    f(\bm{X}) = f \lst{X}
\]
with realizations $\bm{x} = \lst{\bm{x}}$ and $\uncertabs \bm{x} = \lst{\uncertabs \bm{x}}$.

Pick some random values from the normal distribution, then compute the sample variance of the transformed values.

\subsection{Analytical Approach}
This approach systematically computes the pdf of the transformation by use of analytical methods like cdf transformations, convolutions, etc.



\end{document}