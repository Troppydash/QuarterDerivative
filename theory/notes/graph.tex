\documentclass[]{article}
\usepackage{geometry}

\begin{document}
	
\section{Graph Theory}
A graph is a finite set of vertices (nodes) and a set of edges connecting the vertices. Graphs can represent sets of objects with related to each other.

\begin{itemize}
	\item Two vertices are adjacent if there is an edge between them.
	\item Order is the number of vertices in the graph
	\item Size is the number of edges in the graph
	\item Vertex degree is the number of edges connected to the vertex
	\item Isolated vertex refers to a vertex with degree zero
	\item Self loop is an edge from a vertex to itself
	\item A directed graph has all the edges having a direction (start vertex to end vertex)
	\item Undirected graphs have edges that are unidirectional (in both directions)
	\item Weighted graphs have edges with weight values (a number of some sort)
	\item Unweighted graphs have edges with no weight values (implicit 1 valued)
\end{itemize}

\subsection{Breadth first search}
A graph traversing (searching) algorithm, BFS starts at a vertex, explores all its neighbours at a given depth (vertices depth edges away), then increment the depth and explores further.

As graphs can contain cycles, we have to keep track of the visited vertices

Used to find the shortest paths in a graph, find the minimum spanning tree, build indices of web pages by search engine crawlers, search on social networks, and to find available nodes in p2p networks.


\end{document}
